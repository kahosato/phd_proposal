\documentclass[11pt]{article}
\title{Research Proposal}
\author{
 Kaho Sato\\
  \texttt{ks789@cam.ac.uk}
}
\usepackage[round]{natbib}
\usepackage{enumitem}

\begin{document}
\maketitle
\section{Introduction}
As internationalisation progresses at a rapid rate, there is a high global 
interest in learning English, arguably the current lingua franca of the world. 
Application of Natural Language Processing (NLP) techniques to create systems 
that automatically assess learners' writing in English has been an active field 
of research, with the aim of making language learning more accessible. In 
particular, there have been many works on automatic grammatical error detection 
and correction in texts produced by non-native speakers of 
English~\citep{izumi2003automatic, eeg2003automatic, han2006detecting, 
tetreault2008ups, de2008classifier, gamon2008using, gamon2010using, 
ng2014conll}. Especially for learners with low proficiency in English, 
grammatical errors are one of the primary problems that they need to overcome. 
Therefore, automating grammatical error detection and correction has a great 
impact in learning. Grammatical error detection and correction are not only 
useful for the educational purpose, but also can be used to improve writing 
assistance systems which are typically incorporated in word processors like 
Microsoft Word. Catering to the need of non-native speakers who outnumber native 
speakers by a ratio of 3 to 1~\citep{sussex1999david} is commercially 
beneficial, therefore has been stirring up high interest in this particular 
sub-field of NLP. In the following, I exclusively discuss grammatical error 
detection and correction in text produced by \emph{non-native speakers}, whom I 
will refer to as ``authors''.

Specifically, I would like to investigate how the knowledge of authors' first 
language (L1) could improve error detection and correction in their texts 
written in their second language (L2). This idea is based on \emph{Contrastive 
Analysis Hypothesis} proposed by \cite{lado1957linguistics}, which states that 
the difference between one's L1 and L2 determines the difficulty in acquiring L2 
and the errors they make in L2. 
When using L2, one often applies the knowledge from their L1; this phenomenon is 
often referred to as L1 transfer~\citep{wanner1982language, frenck1997syntactic, 
dussias2003syntactic, nitschke2010first}. Though this interference may result in 
a grammatically sound output, it may as well leads to errors. For instance, an 
Italian speaker may say ``*I have 20 years'' instead of ``I am 20 years old'', 
as it is the direct translation of the equivalent expression in Italian. Though 
not all the errors made in L2 can be described by L1 transfer, knowing which 
errors it may caused should improve error detection and correction.

As stated previously, much work has been done in automatic grammatical error 
detection and correction. Errors which have been explored the most are article 
errors~\citep{han2006detecting, rozovskaya2010training, turner2007language, 
dahlmeier2011grammatical, de2008classifier} and preposition 
errors~\citep{bergsma2009web, tetreault2008ups, tetreault2010using, 
dahlmeier2011grammatical, de2008classifier, chodorow2007detection}. This is 
hardly surprising, as they are the most frequent error types and their confusion 
sets are finite since they are closed-class, which makes the error detection and 
correction essentially an n-class classification problem. These errors were the 
main focus of the 2012 HOO Shared Task~\citep{dale2012hoo} and 14 teams have 
presented a system that detects and corrects them.

Collocation errors are also well-studied~\citep{shei2000esl, 
wible2003bootstrapping, futagi2008computational, liu2009automated}, though to a 
lesser extent. This is possibly due to their complexity; the size of confusion 
sets is almost infinite and detection or correction of them can not be treated 
as a classification problem. It instead is necessary to identify potentially 
ambiguous triggers first, and then determining the type of mistake made and 
correcting it. In other words, error correction and detection are two separate 
tasks when we consider open class words errors.

It has been shown that the knowledge on authors' L1 improves algorithms that 
detect and correct errors in the use of prepositions or 
articles~\citep{rozovskaya2010generating, rozovskaya2011algorithm}. There have 
also been similar works for collocation errors~\citep{chang2008automatic, 
dahlmeier2011correcting, shei2000esl, liu2009automated}. However, I believe that 
there is still much room for investigation on how authors' L1 can be used for 
the errors in the use of open-class words for two reasons. Firstly, out of many 
types of errors one can make with open-class words, most works focus on 
verb-noun collocation errors~\citep{chang2008automatic, shei2000esl, 
liu2009automated}. Other errors such as verb form errors or other types of 
collocation errors are yet to be explored. Secondly, most of the works on errors 
with open-class words focus on correction of 
them~\citep{dahlmeier2011correcting, chang2008automatic} and there has been 
little work on L1-specific detection of errors with open-class words.  In my 
research, I would like to carry out a thorough investigation on this very 
promising, yet underexplored approach to automatic grammatical error detection 
and correction.

\section{Proposal}
\subsection{Experiment to motivate the use of L1 in error detection and 
correction}
\label{sub:experiment}
The partial reason why there is not much work that takes authors' L1 into 
consideration is that not in every situation their L1 is known. For instance, 
shared tasks that focuses on grammatical error detection and corrections such as 
the HOO shared tasks and the CoNLL shared tasks do not reveal L1 of the authors 
of the texts in the dataset and it is not likely that word processors such as 
Microsoft Word would know L1 of the user. In my MPhil project, I would like to 
investigate whether a pipeline of L1 identification algorithm and error 
correction algorithm outperforms systems that does not use authors' L1. If it 
does, it makes a very strong case for the use of L1 in error detection and 
correction. The L1 identification task has been investigated by various research 
\citep{estival2007author, tomokiyo2001you, koppel2005determining, 
koppel2005automatically, wong2009contrastive, kochmar2011identification}. 
However, as far as I am aware, there is no work that connects an L1 
identification and error correction algorithms. Clearly, the performance of such 
a pipeline highly depends on the accuracy of the L1 identification algorithm and 
I plan to build on and improve the previous works.

Only a part of L1 that can be observed in Cambridge Learner Corpus is covered by 
the existing works. For instance, \cite{kochmar2011identification} focuses on 
distinguishing texts written by five Romance languages (Italian, Catalan, 
Spanish, Portuguese and French) and five Germanic languages (German, Swiss 
German, Dutch, Swedish and Danish). Cambridge Learner Corpus contains scripts by 
students speaking over 130 different L1. Though it is certainly beyond the scope 
of an MPhil project to cover more L1, let alone all of them. It probably is not 
feasible to cover all 130 languages in a PhD either. However, it might be 
possible and interesting to device an algorithm that identifies the language 
group of L1 of the author, which I would like to investigate as a part of my 
PhD.

\subsection{Improving error detection and correction using L1}
With a strong evidence that supports the benefit of authors' L1 in error 
detection and correction, I want to transition into devising such algorithms.
Among the types of errors non-native speakers may make, I would like to focus on 
detection and correction of the following.
\begin{enumerate}
\item Errors in collocations of arbitrary syntactic categories\label{error:1}
\item Verb form errors \label{error:2}
\item Spelling Errors \label{error:3}
\end{enumerate}
In the following sub-sections, I further describe and specify each error and the 
approach I would like to take to tackle it.
\subsubsection{Collocation errors}
\label{subsub:collocation}
As stated previously, most works use authors' L1 to \emph{correct} errors in 
verb-noun collocation. This is only one kind among six, which are:
\begin{itemize}
\item verb + noun
\item adjective + noun
\item noun + noun
\item adverb + adjective
\item verb + prepositional phrase
\item verb + adverb
\end{itemize}
All collocations are hugely language dependent and it is easy to imagine how 
one's L1 can interfere to cause errors. In my PhD, I would like to work on both 
detection and correction of each type listed above.

I would like to extend the approach presented in the work by 
\cite{kochmarcross}.  This work is distinct from other works on collocation 
error detection and correction that use the authors’ L1 for two reasons. 
Firstly, it is an algorithm for detection, rather than correction, of errors in 
verb-noun collocation errors. Secondly, rather than using L1-specific properties 
such as the error patterns of speakers of a given L1, it utilises a semantic 
model of L1 and L2. These semantic models are built from corpus statistics in 
each language independently and errors of lexical choice is done using the 
discrepancies between the two models. The results are promising, and it also 
shows that the semantic model learned from a particular L1 is portable to other, 
typologically related languages. This approach was applied to texts produced by 
Spanish and Russian native speakers. I would like to further investigate whether 
this algorithm can be used to detect other types of errors in texts produced by 
not only Spanish or Russian speakers, but also other L1 speakers on which data 
is available in the Cambridge Learner Corpus.
					
Most works in collocation error correction have relied on dictionaries or 
manually created databases~\citep{shei2000esl, wible2003bootstrapping, 
futagi2008computational}. Dictionaries suffer from low coverage and does not 
provide probability estimate for each candidate translation, and manually 
creating databases is expensive and not scalable. I would like to investigate 
more data-driven methods, taking advantage of the Cambridge Learner Corpus. For 
instance, I plan to investigate whether the confusion matrix for a particular L1 
estimated from the actual learner corpus can be used for collocation errors of 
arbitrary syntactic types.

\subsubsection{Verb form errors}
\label{subsub:verb}
A subset of verb form errors that require little contextual information has been 
investigated well, such as improper user of modals (\emph{they can *built/build 
a new house}) and overregularised verb inflection 
(*goed/went)~\citep{chodorow2000unsupervised, leacock2003automated}. These 
errors are handled well by simple rule-based approaches and they are rarely 
result of L1 transfer, but simply of author's lack of L1 knowledge. The verb 
form error that I would like to focus on is verb tense errors. Some tense 
systems are simpler than that of English and some are more complex. For 
instance, Chinese is a tenseless language in which the temporal information is 
conveyed by words that accompany the verb and Italian has finer distinctions in 
the past tense than English. It is expected that there is a difference in the 
error pattern between authors of different L1 and therefore L1 could be a useful 
clue in identifying this type of errors. Verb tense errors are more challenging 
to detect compared to, for example, overregularised verb inflection, because 
tenses may be influenced by factors that are not local to the sentence. For 
example, ``I have a plate of pasta.'' alone is a perfectly grammatical sentence. 
However, when combined with another sentence in past tense, like in ``I had a 
great dinner last night. I *have a plate of pasta.'', it is most likely not in 
the tense that the author intended.

\cite{tajiri2012tense} regarded verb tense detection and correction as a 
sequence labelling task, where each member is a verb that appears in the 
document. The set of features that describe a verb includes namely its 
part-of-speech tag, its arguments such as subject and object. I would like to 
augment the model with the authors' L1 and train it on Cambridge Learner Corpus 
and see whether the performance improves.

It would also be interesting to make an empirical study of the pattern of verb 
tense errors with respect to authors' L1. The outcome of such study can 
potentially be used to extend the work by \cite{lee2008correcting}. In this 
work, errors are detected by template matching on parse trees. The templates are 
created from parse trees that represents a sentence to which they introduced a 
likely error that a non-native speaker would make. Since the sentence is 
ungrammatical, the parsed tree should look different from those of grammatical 
sentences. Since the errors which an author is likely to make partially depend 
on their L1, we could create different templates for each L1 using the error 
patterns found in the empirical study.

\subsubsection{Spelling Errors}
Spelling errors can be committed by both native and non-native speakers. Task of 
detecting and correcting spelling errors has been extensively studied for much 
longer time than any other errors~\citep{heift2007errors}. However, the majority 
of the work investigates errors made by native speakers. Spelling errors made by 
non-native speakers are very different from those made by native speakers. The 
former tend to be a multiple edit distance away from the intended word, where 
the latter tend to have a single edit distance. For example, a native speaker 
might write ``*spellinf'' instead of ``spelling'', because keys for f and g are 
adjacent to each other in the QWERTY layout. This spelling mistake is single 
edit distance away from the correct spelling, as ``spelling'' can be obtained 
simply from substituting ``f'' with ``g''. Though this type of mistake can also 
be made by non-native speakers, some of the spelling mistakes non-native 
speakers make rarely occur in texts produced by a native speaker. An example of 
such mistakes made by Japanese native speakers is ``*libulary''. This is two 
edit distance away from ``library'', as it can be recovered by eliminating ``u'' 
and substituting the second ``l'' with ``r''. 

Because of the difference in the nature, many spelling checkers perform terribly 
on non-native speakers texts. Indeed, \cite{rimrott2005language} demonstrates 
that only 48\% of spelling errors in German texts produced by non-native 
speakers were corrected by Microsoft Word 2003 spelling checker, precisely 
because they are designed to catch native speakers' spelling errors. This shows 
that spelling errors of non-native speakers must be dealt in a different way. 

\cite{rimrott2008evaluating} claims that spelling errors made by non-native 
speakers can be classified into three categories which are lexical, 
morphological and phonological errors. Lexical errors are errors that are made 
because of the author simply did not know the word in L2. This is often easy to 
detect as it often results to a non-word spelling error. Morphological errors 
are characterised by incorrect inflection. For instance, one may not know that 
``go'' is a irregular verb and write ``goed'', overgeneralising the 
morphological rules. This type of errors is not made due to L1 transfer, but 
should be attributed to the author's lack of the knowledge. Phonological errors 
are caused by the actual or assumed phonology. The earlier example of 
``*libulary/library'' is a phonological error. In the Japanese language, with 
few exceptions, each consonance is followed by a vowel, which caused an 
unnecessary ``u'' after ``b''.  The substitution of ``l'' with ``r'' is another 
very common mistake among Japanese native speakers, because ``l'' and ``r'' are 
not distinct phonemes in Japanese unlike in English. Such an error pattern must 
vary from author to author, depending on their L1, therefore, detection and 
correction of phonological errors should greatly benefit from knowing the 
authors' L1.

One may take a completely statistical approach to detect phonological errors. 
However, the data might be very sparse. In general, the difficulty of 
statistical error detection and correction lies in the fact that misuse of the 
language is much less common that its proper use. It is easy to imagine that a 
purely statistical approach to detect a sub-type of spelling errors will not 
work well, because of the lack of enough data. Thus It might be better to 
extract a common phonological error patterns from Cambridge Learner Corpus and 
create heuristic rules for detection and correction by generalising them.

\section{Work plan}
To conclude, I present my work plan in the first year in Table 
\ref{tab:workplan}. Admittedly, not everything discussed in this proposal is 
included in the plan. The remaining will be worked on in the following years. It 
is also expected that more research questions will spawn from the first year of 
research, which will determine the future work.
\begin{table}[]
\centering
\caption{Work plan}
\label{tab:workplan}
\begin{tabular}{|c|l|}
\hline
Oct-Dec   & Extension of MPhil project, as described in Section 
\ref{sub:experiment}.                        \\ \hline
Jan-Mar   & \begin{tabular}[l]{@{}l@{}}Investigate verb + noun, adjective + noun 
and noun + noun\\ collocation errors, as described in Section 
\ref{subsub:collocation}.\end{tabular}\\ \hline
Apr-Jun   & \begin{tabular}[l]{@{}l@{}}Investigate adverb + adjective, verb + 
prepositional phrase\\ and verb + adverb collocation errors, as described in 
Section \ref{subsub:collocation} \end{tabular}\\ \hline
Jul-Sep   & Investigate verb form errors, as described in Section 
\ref{subsub:verb}                                  \\ \hline
\end{tabular}
\end{table}

% L1 transfer is undeniably the source of many errors made by non-native 
speakers. Previous works have shown that L1 of the learner can improve error 
correction in texts produced by language learners. I would like to build on 
these works, namely by investigating other types of errors, making the algorithm 
more data-driven, extending it to error correction and using finer grained 
grouping of L1 languages. 



(2525 words)



\bibliography{bib}
\bibliographystyle{plainnat}
\end{document}

